\paragraph{Essential properties and advantages/disadvantages}
The first model which we implement is the probabilistic PCA model introduced in Chapter 12.2.1 of \citep{bishop2006pattern}. The reason this model is attractively is primarily due to the fact that its likelihood is tractable. In particular, given a dataset $\bm{X} = \lrc{\bm{x}_n}$ of $N$ samples $\bm{x}_n$ with $D$ dimensions each, the log likelihood of the PPCA model is given by $\ln p(\bm{X}|\bm{W}, \bm{u}, \sigma^2) = -\frac{N}{2}\lr{D \ln(2\pi) + \ln \lra{\bm{C}} + \text{Tr}\lr{\bm{C}^{-1}\bm{S}}}$, where $\bm{C} = \bm{W}\bm{W}^T + \sigma^2 \bm{I}$ and $\bm{S}$ is the sample covariance matrix of $\bm{X}$. Moreover, there exists a closed form-solution for the parameters $\bm{W}, \bm{u}, \sigma^2$, which maximize this log-likelihood. 

The existence of a closed-form expression for the likelihood function is useful, as it allows for direct comparison with other density model. The fact that its the maximizing parameters also have a closed-form means we can easily optimize the PPCA model. At the same time, the PPCA model also allows for numerical optimization using the Expectation-Maximization (EM) algorithm, which can be more efficient in higher dimensions and when dealing with missing data. From a practical point of view, the PPCA model is useful, as it can be used both for dimensionality-reduction (like the regular PCA model), but also for random sampling due to its probabilistic nature. Finally, a major advantage of the PPCA model is that it allows capturing the most important covariance in its included principal components, while averaging the variance for all the remaining left-out principal components in its parameter $\sigma^2$. 

Perhaps the most obvious disadvantage of the PPCA model is that it is based on a linear-gaussian framework and hence, unlike the VAE, may not be able to capture complex non-linear structure in the data it is trained on. On another note, it is worth pointing out that the maximum-likelihood solution to the PPCA model only determines the latent space up to an arbitrary rotation $\bm{R}$. We shall for simplicity set $\bm{R} = \bm{I}$. Moreover, we shall use $M=2$ principal components, so that we may easily compare the latent space of the PPCA model with that of the previously introduced VAE models.