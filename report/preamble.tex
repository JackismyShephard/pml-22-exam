%\documentclass[11pt,a4paper]{article}
\documentclass{article}
\usepackage{arxiv}

%%%%%%% META STUFF %%%%%%%%%%%%%%%
\usepackage{xifthen}
\usepackage{xparse}
\usepackage{lipsum}
%%%%%%%%FONT%%%%%%%%%%%%%%%%%%%%%%
\usepackage[utf8]{inputenc}
\usepackage[T1]{fontenc}    % use 8-bit T1 fonts
\usepackage{microtype}
\usepackage{upgreek}
%\usepackage{lmodern}
%\usepackage[english]{babel}

%%%%%%%LAYOUT%%%%%%%%%%%%%%%%%%%%%%%%
%\usepackage{a4wide}
\usepackage[hang]{footmisc}
\usepackage[toc,page]{appendix}
%\usepackage[nofoot, left=2.20cm, bottom=1.5cm, right=2.20cm, top=2.5cm]{geometry}
%\usepackage{parskip} % automatic line breaks
\usepackage{enumitem}
\usepackage{subcaption}
\usepackage{tabularx}
\usepackage{longtable}
\usepackage{floatrow}
\usepackage{array}
\newcolumntype{|}{!{\vrule width 0.5pt}}
\newfloatcommand{capbtabbox}{table}[][\FBwidth]
\usepackage{booktabs}

%%%%%GRAPHICS%%%%%%%%%%%%%%%%%%%%%%%%%%%
\usepackage{float} % https://ctan.org/pkg/float
\usepackage{graphicx}  % For including images
\graphicspath{{../code/figures/}}
% do not move figures across sections
\usepackage[section]{placeins}

%%%%%%%%% color %%%%%%%%%%%%%%%%%
\usepackage{xcolor}
\definecolor{mygray}{HTML}{F7F7F7}

%%%%%%%%%%||bibliography||%%%%%%%%%%%%%
\usepackage{natbib}
\usepackage[nottoc,numbib]{tocbibind}
\usepackage{doi}
\nocite{*}
%\usepackage[maxbibnames=4,backend=biber,style=alphabetic, autocite=footnote]{biblatex}
%\usepackage{apacite} % https://ctan.org/pkg/apacite
%\setlength\bibitemsep{0.4\baselineskip}
%\DeclareFieldFormat{url}{URL\addcolon\space\url{#1}}
%\DeclareFieldFormat{isbn}{ISBN\addcolon\space{#1}}
%\DeclareFieldFormat{doi}{%
%    DOI\addcolon\space
%    \ifhyperref
%    {\href{https://doi.org/#1}{\nolinkurl{#1}}}
%    {\nolinkurl{#1}}%
%}

%%%%%%MATH%%%%%%%%%%%%%%%%%%%%%%%%%%%
%\usepackage{amsfonts}
%\usepackage{amsmath}
\usepackage{mathtools}
\usepackage{amssymb}
\usepackage{latexsym}
\usepackage{dsfont}
\usepackage{nicefrac}       % compact symbols for 1/2, etc.
\usepackage{bm}
\usepackage{siunitx}

%\usepackage{mathpazo}
%\allowdisplaybreaks

\newcommand{\abs}[1]{\lvert #1 \rvert}
\newcommand{\norm}[1]{\left\lVert#1\right\rVert}
\DeclareMathOperator*{\argmax}{arg\,max}
\DeclareMathOperator*{\argmin}{arg\,min}
% Left-right bracket
\newcommand{\lr}[1]{\left (#1\right)}
% Left-right square bracket
\newcommand{\lrs}[1]{\left [#1 \right]}
% Left-right curly bracket
\newcommand{\lrc}[1]{\left \{#1\right\}}
% Left-right absolute value
\newcommand{\lra}[1]{\left |#1\right|}
% Left-right upper value
\newcommand{\lru}[1]{\left \lceil#1\right\rceil}
% Scalar product
\newcommand{\vp}[2]{\left \langle #1 , #2 \right \rangle}
% The real numbers
\newcommand{\R}{\mathbb R}
% The natural numbers
\newcommand{\N}{\mathbb N}
% Expectation symbol with an optional argument
\NewDocumentCommand{\E}{o}{\mathbb E\IfValueT{#1}{\lrs{#1}}}
% Indicator function with an optional argument
\NewDocumentCommand{\1}{o}{\mathds 1{\IfValueT{#1}{\lr{#1}}}}
% Probability function
\let\P\undefined
\NewDocumentCommand{\P}{o}{\mathbb P{\IfValueT{#1}{\lr{#1}}}}
% A hypothesis space
\newcommand{\HH}{\mathcal H}
% A sample space
\newcommand{\XX}{\mathcal{X}}
% A label space
\newcommand{\YY}{\mathcal{Y}}
% A nicer emptyset symbol
\newcommand{\FF}{\mathcal{F}}
\let\emptyset\varnothing
% Sign operator
\DeclareMathOperator{\sign}{sign}
\newcommand{\sgn}[1]{\sign\lr{#1}}
% KL operator
\DeclareMathOperator{\KL}{KL}
% kl operator
\DeclareMathOperator{\kl}{kl}
% The entropy
\let\H\relax
\DeclareMathOperator{\H}{H}
% Majority vote
\DeclareMathOperator{\MV}{MV}
% Variance
\DeclareMathOperator{\V}{Var}
\NewDocumentCommand{\Var}{o}{\V\IfValueT{#1}{\lrs{#1}}}
% VC
\DeclareMathOperator{\VC}{VC}
% VC-dimension
\newcommand{\dVC}{d_{\VC}}
% FAT ...
\DeclareMathOperator{\FAT}{FAT}
\newcommand{\dfat}{d_{\FAT}}
\newcommand{\lfat}{\ell_{\FAT}}
\newcommand{\Lfat}{L_{\FAT}}
\newcommand{\hatLfat}{\hat L_{\FAT}}
% Distance
\DeclareMathOperator{\dist}{dist}
\newcommand{\x}{\bm{x}}
\newcommand{\y}{\bm{y}}
\newcommand{\w}{\bm{w}}
\newcommand{\T}{\text{T}}
\newcommand{\U}{\bm{U}}
\newcommand{\bu}{\bm{u}}

\def\vec#1{\mathchoice{\mbox{\boldmath$\displaystyle#1$}}
    {\mbox{\boldmath$\textstyle#1$}}
    {\mbox{\boldmath$\scriptstyle#1$}}
    {\mbox{\boldmath$\scriptscriptstyle#1$}}}

%%%%%%% Algorithms and theorems %%%%%%%%%%%%
\usepackage[ruled, linesnumbered]{algorithm2e}
\usepackage{amsthm}
\theoremstyle{theorem}
\newtheorem{theorem}{Theorem}
\theoremstyle{lemma}
\newtheorem{lemma}{Lemma}
\theoremstyle{corollary}
\newtheorem{corollary}{Corollary}


%%%%%%%%%||links||%%%%%%%%%%
\usepackage{url}
\urlstyle{rm}
\usepackage{hyperref}
\usepackage[all]{hypcap}
%\hypersetup{
%    colorlinks,
%    linkcolor={red!50!black},
%    citecolor={blue!50!black},
%    urlcolor={blue!80!black}
%}
\usepackage[noabbrev, capitalise, nameinlink]{cleveref}
\crefname{lstlisting}{listing}{listings}
\Crefname{lstlisting}{Listing}{Listings}


%%%%%%%CODE%%%%%%%
%\usepackage[outputdir=doc]{minted}
%\setminted{fontsize = \small, breaklines=true, bgcolor=mygray}
%\AtBeginEnvironment{listing}{\setcounter{listing}{\value{lstlisting}}}
%\AtEndEnvironment{listing}{\stepcounter{lstlisting}}
%\newenvironment{code}{\captionsetup{type=listing}}{}
%\AtBeginEnvironment{code}{\setcounter{listing}{\value{lstlisting}}}
%\AtEndEnvironment{code}{\stepcounter{lstlisting}}
%\renewcommand{\theFancyVerbLine}{\rmfamily {\footnotesize{\arabic{FancyVerbLine}}}}
\usepackage{listings}  % For presenting code 
% Sensible defaults for lstlistings
\lstdefinestyle{mystyle}{
language=Python,
basicstyle=\ttfamily\footnotesize,
backgroundcolor=\color[HTML]{F7F7F7},
rulecolor=\color[HTML]{EEEEEE},
identifierstyle=\color[HTML]{24292E},
emphstyle=\color[HTML]{005CC5},
keywordstyle=\color[HTML]{D73A49},
commentstyle=\color[HTML]{6A737D},
stringstyle=\color[HTML]{032F62},
emph={@property,self,range,True,False},
morekeywords={super,with,as,lambda},
literate=%
    {+}{{{\color[HTML]{D73A49}+}}}1
{-}{{{\color[HTML]{D73A49}-}}}1
{*}{{{\color[HTML]{D73A49}*}}}1
{/}{{{\color[HTML]{D73A49}/}}}1
{=}{{{\color[HTML]{D73A49}=}}}1
{/=}{{{\color[HTML]{D73A49}=}}}1,
breaklines=true,
showstringspaces=false,
frame=L,
belowcaptionskip=1\baselineskip,
xleftmargin=\parindent
}
\lstset{style=mystyle}